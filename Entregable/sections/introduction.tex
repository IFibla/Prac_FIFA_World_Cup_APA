\section{Introdución}

El Mundial de Fútbol es un torneo internacional de fútbol que se celebra cada cuatro años y es organizado por la Federación Internacional de Fútbol Asociación (FIFA). Es considerado el evento deportivo más importante del mundo y cuenta con la participación de las selecciones nacionales de fútbol de todos los países afiliados a la FIFA.
\newline

El primer Mundial de Fútbol se celebró en 1930 en Uruguay y desde entonces se ha realizado cada cuatro años, excepto durante la Segunda Guerra Mundial, cuando no se disputó el torneo de 1942 y 1946. Actualmente, el Mundial de Fútbol consta de una fase de grupos y una fase eliminatoria, y culmina con la final, que determina al campeón del mundo. La selección que resulte campeona del Mundial de Fútbol recibe la Copa del Mundo, un trofeo que se entrega al ganador del torneo.
\newline

Este año se celebrará en Qatar y será la 22ª edición de este importante evento deportivo. En este trabajo, mediante el uso del aprendizaje automático, se analizarán diversos factores que pueden influir en el rendimiento de las selecciones participantes y se realizarán predicciones sobre quiénes podrían ser los favoritos para ganar el torneo.
\newline

Nuestros modelos de aprendizaje automático, van a ajustar una recta de regresión que nos va a permitir simular y predecir los resultados de todos los encuentros del Mundial. Un partido de futbol tiene muchos resultados posibles, pero se podrían resumir en Victoria, que puntúa 3 puntos, Derrota, que no aporta puntos, o Empate, que aporta un punto a cada equipo. Como el Mundial consta de una fase de grupos con un enfrentamiento por pareja de equipos, se ha visto que la posibilidad de que dos equipos empaten a puntos en esta fase es bastante elevada, por eso, se ha decidido tomar como variables objetivos los goles marcados por ambos equipos, una decisión, que nos permitirá mantener la diferencia de goles y así, clasificar a los equipos correctamente.
\newpage