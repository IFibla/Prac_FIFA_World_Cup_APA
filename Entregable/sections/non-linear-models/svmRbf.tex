\subsection{Máquinas de vectores de soporte}
Las \textit{Support Vector Machines} se usan más frecuentemente para resolver problemas de clasificación aunque con pequeñas adaptaciones se pueden usar para resolver problemas de regressión.
\newline

El estudio de este modelo ha empezado explorando los hiperparámetros para saber que combinación da los mejores resultados. En nuestro caso exploramos los siguientes hiperparámetros: \texttt{'C'} y \texttt{'gamma'}. Después de entrenar varios modelos con diferentes hiperparámetros nos hemos encontrado con el mejor, siendo teniendo los siguientes hiperparámetros: \texttt{1.9952} y \texttt{auto} respectivamente con un acierto con una media del 0,285.
\newline

Podemos ver un acierto con una media mayor al modelo anterior y esto puede deberse al hecho de que con el truco del kernel lo que hacemos es transformar un modelo no lineal a uno lineal con más dimensiones y esto puede suponer un problema para problemas como este cuando se trata de hacer regresiones.

